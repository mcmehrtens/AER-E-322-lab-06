% AER E 361 Mission Report Template
% Spring 2023
% Template created by Yiqi Liang and Professor Matthew Nelson

% Document Configuration DO NOT CHANGE
\documentclass[12 pt]{article}
% --------------------LaTeX Packages---------------------------------
% The following are packages that are used in this report.
% DO NOT CHANGE ANY OF THE FOLLOWING OR YOUR REPORT WILL NOT COMPILE
% -------------------------------------------------------------------

\usepackage{hyperref}
\usepackage{parskip}
\usepackage{titlesec}
\usepackage{titling}
\usepackage{graphicx}
\usepackage{graphviz}
\usepackage[T1]{fontenc}
\usepackage{titlesec, blindtext, color} %for LessIsMore style
\usepackage{tcolorbox} %for references box
\usepackage[hmargin=1in,vmargin=1in]{geometry} % use 1 inch margins
\usepackage{float}
\usepackage{tikz}
\usepackage{svg} % Allows for SVG Vector graphics
\usepackage{textcomp, gensymb} %for degree symbol
\hypersetup{
	colorlinks=true,
	linkcolor=blue,
	urlcolor=cyan,
}
\usepackage{biblatex}
\addbibresource{lab-report-bib.bib}
\usepackage{amsmath}
\usepackage{upgreek}
\usepackage{listings}
\usepackage{multicol}
\usepackage{array}

\usepackage{hologo} %KYR: for \BibTeX
%\usepackage{algpseudocode}
%\usepackage{algorithm}
% This configures items for code listings in the document
\usepackage{xcolor}

\usepackage{fancyhdr} % Headers/Footers
\usepackage{siunitx} % SI units
\usepackage{csquotes} % Display Quote
\usepackage{microtype} % Better line breaks

\definecolor{commentsColor}{rgb}{0.497495, 0.497587, 0.497464}
\definecolor{keywordsColor}{rgb}{0.000000, 0.000000, 0.635294}
\definecolor{stringColor}{rgb}{0.558215, 0.000000, 0.135316}
\definecolor{mygreen}{rgb}{0,0.6,0}
\definecolor{mygray}{rgb}{0.5,0.5,0.5}
\definecolor{mymauve}{rgb}{0.58,0,0.82}

\lstdefinestyle{customc}{
  belowcaptionskip=1\baselineskip,
  breaklines=true,
  frame=L,
  xleftmargin=\parindent,
  language=C,
  showstringspaces=false,
  basicstyle=\footnotesize\ttfamily,
  keywordstyle=\bfseries\color{green!40!black},
  commentstyle=\itshape\color{purple!40!black},
  identifierstyle=\color{blue},
  stringstyle=\color{orange},
 }

 \lstset{ %
  backgroundcolor=\color{white},   % choose the background color; you must add \usepackage{color} or \usepackage{xcolor}
  basicstyle=\footnotesize,        % the size of the fonts that are used for the code
  breakatwhitespace=false,         % sets if automatic breaks should only happen at whitespace
  breaklines=true,                 % sets automatic line breaking
  captionpos=b,                    % sets the caption-position to bottom
  commentstyle=\color{commentsColor}\textit,    % comment style
  deletekeywords={...},            % if you want to delete keywords from the given language
  escapeinside={\%*}{*)},          % if you want to add LaTeX within your code
  extendedchars=true,              % lets you use non-ASCII characters; for 8-bits encodings only, does not work with UTF-8
  frame=tb,	                   	   % adds a frame around the code
  keepspaces=true,                 % keeps spaces in text, useful for keeping indentation of code (possibly needs columns=flexible)
  keywordstyle=\color{keywordsColor}\bfseries,       % keyword style
  language=Python,                 % the language of the code (can be overrided per snippet)
  otherkeywords={*,...},           % if you want to add more keywords to the set
  numbers=left,                    % where to put the line-numbers; possible values are (none, left, right)
  numbersep=8pt,                   % how far the line-numbers are from the code
  numberstyle=\tiny\color{commentsColor}, % the style that is used for the line-numbers
  rulecolor=\color{black},         % if not set, the frame-color may be changed on line-breaks within not-black text (e.g. comments (green here))
  showspaces=false,                % show spaces everywhere adding particular underscores; it overrides 'showstringspaces'
  showstringspaces=false,          % underline spaces within strings only
  showtabs=false,                  % show tabs within strings adding particular underscores
  stepnumber=1,                    % the step between two line-numbers. If it's 1, each line will be numbered
  stringstyle=\color{stringColor}, % string literal style
  tabsize=2,	                   % sets default tabsize to 2 spaces
  title=\lstname,                  % show the filename of files included with \lstinputlisting; also try caption instead of title
  columns=fixed                    % Using fixed column width (for e.g. nice alignment)
}

\lstdefinestyle{customasm}{
  belowcaptionskip=1\baselineskip,
  frame=L,
  xleftmargin=\parindent,
  language=[x86masm]Assembler,
  basicstyle=\footnotesize\ttfamily,
  commentstyle=\itshape\color{purple!40!black},
}

\lstset{escapechar=@,style=customc}

\titlelabel{\thetitle.\quad}

% From here on out you can start editing your document
\newcommand{\subtitle}[1]{%
  \posttitle{%
    \par\end{center}
    \begin{center}\LARGE#1\end{center}
    \vskip0.5em}%
}

\title{\textbf{Iowa State University
\\{\Large Aerospace Engineering}}}
\subtitle{AER E 322 Lab 6\\
		  Composite Laminate Design}
\author{Matthew Mehrtens, Peter Mikolitis, and Natsuki Oda}

\newcommand{\etal}{\textit{et al}., }
\newcommand{\ie}{\textit{i}.\textit{e}., }
\newcommand{\eg}{\textit{e}.\textit{g}., }

% Define the headers and footers
\setlength{\headheight}{70.63135pt}
\geometry{head=70.63135pt, includehead=true, includefoot=true}
\pagestyle{fancy}
\fancyhead{}\fancyfoot{} % clears the headers/footers
\fancyhead[L]{\textbf{AER E 322}}
\fancyhead[C]{\textbf{Aerospace Structures Laboratory Summary}\\
			  \textbf{Lab 6 Composite Laminate Design}\\
			  Section 4 Group 2\\
			  Matthew Mehrtens, Peter Mikolitis, and Natsuki Oda\\
			  \today}
\fancyhead[R]{\textbf{Spring 2023}}
\fancyfoot[C]{\thepage}

\begin{document}
\maketitle
\tableofcontents
\section{Introduction} \label{introduction}
In this lab, we aimed to design a composite layup pattern with the maximum resistance to bending by utilizing the lecture notes and the online composite calculator. Firstly, we studied the lecture notes carefully to gain a better understanding of composite materials and their properties. Next, we experimented with the online composite calculator to design a layup pattern that would maximize the resistance to bending. Through a series of five or more trials, we arrived at what we believe is the ultimate value of $\kappa_{xy}$, the twisting curvature.

\section{Objectives} \label{objectives}
The objectives of this lab experiment were to:

\begin{enumerate}
	\item Understand the concepts and theory presented in the week seven and eight lectures regarding composite materials.
	\item Analyze how the twisting curvature, $\kappa_{xy}$, changes based on alterations to the layup pattern.
	\item Utilize an online composite calculator to experimentally determine the maximum value of $\kappa_{xy}$. The calculator was used to simulate different layup patterns and determine the maximum twisting curvature values that could be achieved under given material properties.
	\item Find the maximum value of $\kappa_{xy}$ given the default online calculator configuration. By using the online composite calculator and experimentally determining the twisting curvature values for various layup patterns, we aimed to identify the maximum twisting curvature value achievable for a given set of material properties.
\end{enumerate}

\section{Hypothesis} \label{hypothesis}
We predict that a layup design with \numrange{2}{3} layers in between \ang{30} and \ang{60} will result in a maximum absolute twisting curvature. We suspect layup angles at \ang{0} and \ang{90} will have at best no effect and at worst a negative effect on the twisting curvature.

\section{Work Assignments} \label{work_assignments}
Refer to Table \ref{table:work_assignments} for the distribution of work during this lab.

\begin{table}[!htbp]
\caption{Work assignments for AER E 322 Lab 6.}
\begin{center}
	\begin{tabular}{| c | c | c | c |}
		\hline
		\multicolumn{1}{| c |}{\textbf{Task}} & \textbf{Matthew} & \textbf{Peter} & \textbf{Natsuki} \\
		\hline
		\multicolumn{4}{| c |}{\textit{Lab Work}} \\
		\hline
		Date Recording & X & & \\
		\hline
		Exp. Setup & X & X & X \\
		\hline
		Exp. Work & X & X & X \\
		\hline
		\multicolumn{4}{| c |}{\textit{Report}} \\
		\hline
		Introduction & X & & X \\
		\hline
		Objectives & X & X & \\
		\hline
		Hypothesis & & & X \\
		\hline
		Materials & & X & \\
		\hline
		Apparatus & & X & \\
		\hline
		Procedures & & & X \\
		\hline
		Data & X & & \\
		\hline
		Analysis & X & X & X \\
		\hline
		Conclusion & X & & \\
		\hline
		Editing & X & & \\
		\hline
	\end{tabular}
\end{center}
\label{table:work_assignments}
\end{table}

\section{Materials} \label{materials}
\begin{itemize}
	\item Week seven and eight lecture notes
	\item Online composite calculator
\end{itemize}

\section{Apparatus} \label{apparatus}
The simulated beam in this lab was a cantilevered, metal \qtyproduct{6x1}{inch} beam. A simulated torsional bending moment was applied to the free end of the beam.

\section{Procedures} \label{procedures}
First, we reviewed the week seven and eight lecture notes individually and as a group to learn the theory of laminate composites. After loading the calculator, we set the following parameters:
\begin{align*}
	t_\mathrm{layer}&=\qty{150}{\micro\meter}\\
	\sigma_x&=0\\
	M_x&=\qty{1}{N.m}
\end{align*}
After these values were set, we experimented with different layup configurations to observe how each configuration effected the twisting curvature, $\kappa_{xy}$.  

\section{Data} \label{data}
The value of $\kappa_{xy}$ from our five designs in shown in Table \ref{tbl:data}.

\begin{table}[!htbp]
\caption{Values of $\kappa_{xy}$ with respect to each of the different layup configurations.}
\begin{center}
	\begin{tabular}{| c | c | c | c |}
		\hline
		Design&Layup Pattern&$M_x$ [\unit{N.m}]&$\lvert\kappa_{xy}\rvert$ [\unit{m^{-1}}]\\
		\hline
		\num{1}&\num{0}/\num{90}&1&\num{1.37e-17}\\
		\hline
		\num{2}&\num{0}/\num{30}/\num{60}/\num{90}&1&\num{0.0761}\\
		\hline
		\num{3}&\num{0}/\num{45}/\num{90}&1&\num{0.0864}\\
		\hline
		\num{4}&$\pm$\num{45}&1&\num{28.3}\\
		\hline
		\num{5}&$\pm$\num{27}&1&\num{46.6}\\
		\hline
	\end{tabular}
\end{center}
\label{tbl:data}
\end{table}

\section{Analysis} \label{analysis}
\textbf{Laminate Analysis}

We are given in the lecture notes that
\begin{align} \label{eqn:b}
	\textbf{B}&=\frac{1}{2}\sum_{i=1}^K\overline{\textbf{S}}_i(z_{i+1}^2-z_i^2)
\end{align}
where \textbf{B} is the coupling stiffness matrix, $K$ is the number of laminate layers, $\overline{\textbf{S}}_i$ is the stiffness matrix for the $i$th layer, and $z$ is the distance from the midplane to the surface of a layer. Positive $z$ is above the midplane and negative $z$ is below the midplane. The top layer of the composite ($i=1$) is bounded above by $z_1$ and below by $z_2$, the second layer from the top ($i=2$) is bounded above by $z_2$ and below by $z_3$, and the bottom layer of the composite ($i=K$) is bounded above by $z_K$ and below by $z_{K+1}$. In general, the $i$th layer is bounded above by $z_i$ and below by $z_{i+1}$. We must show that for a symmetrical laminate layup, \textbf{B} is zero. In order for \textbf{B} to be zero, the summation from Equation \ref{eqn:b}
\[\sum_{i=1}^K\overline{\textbf{S}}_i(z_{i+1}^2-z_i^2)\]
must be zero. We begin the proof by expanding the summation as shown below.
\begin{align*}
	\sum_{i=1}^K\overline{\textbf{S}}_i(z_{i+1}^2-z_i^2)&=\overline{\textbf{S}}_1(z_2^2-z_1^2)+\overline{\textbf{S}}_2(z_3^2-z_2^2)+\cdots+\overline{\textbf{S}}_{K-1}(z_K^2-z_{K-1}^2)+\overline{\textbf{S}}_K(z_{K+1}^2-z_K^2)\\
	&=\overline{\textbf{S}}_1z_2^2-\overline{\textbf{S}}_1z_1^2+\overline{\textbf{S}}_2z_3^2-\overline{\textbf{S}}_2z_2^2+\cdots\\
	&\quad+\overline{\textbf{S}}_{K-1}z_K^2-\overline{\textbf{S}}_{K-1}z_{K-1}^2+\overline{\textbf{S}}_Kz_{K+1}^2-\overline{\textbf{S}}_Kz_K^2
\end{align*}
Since the composite is symmetrical, we assume $\overline{\textbf{S}}_1=\overline{\textbf{S}}_K$, $\overline{\textbf{S}}_2=\overline{\textbf{S}}_{K-1},\cdots$. Additionally, due to the symmetry and according to our definition of $z$, we know that $z_1=-z_{K+1}$, $z_2=-z_{K},\cdots$. It follows that $z_1^2=z_{K+1}^2$, $z_2^2=z_K^2,\cdots$. We then come up with the following equivalencies:
\begin{align*}
	\overline{\textbf{S}}_1z_1^2&=\overline{\textbf{S}}_Kz_{K+1}^2\\
	\overline{\textbf{S}}_1z_2^2&=\overline{\textbf{S}}_Kz_K^2\\
	\overline{\textbf{S}}_2z_2^2&=\overline{\textbf{S}}_{K-1}z_K^2\\
	\overline{\textbf{S}}_2z_3^2&=\overline{\textbf{S}}_{K-1}z_{K-1}^2
\end{align*}
We now continue the expansion of the summation below substituting the $\overline{\textbf{S}}_K$ and $\overline{\textbf{S}}_{K-1}$ terms with their equivalent $\overline{\textbf{S}}_1$ and $\overline{\textbf{S}}_2$ terms:
\begin{align*}
	\sum_{i=1}^K\overline{\textbf{S}}_i(z_{i+1}^2-z_i^2)&=\overline{\textbf{S}}_1z_2^2-\overline{\textbf{S}}_1z_1^2+\overline{\textbf{S}}_2z_3^2-\overline{\textbf{S}}_2z_2^2+\cdots+\overline{\textbf{S}}_2z_2^2-\overline{\textbf{S}}_2z_3^2+\overline{\textbf{S}}_1z_1^2-\overline{\textbf{S}}_1z_2^2\\
	&=0
\end{align*}
As shown above, if the composite is symmetric, all the terms in the summation will cancel out. Therefore, in a symmetric composite, the coupling stiffness matrix, \textbf{B}, is zero.

For a composite material, we are given the following matrix equation, shown in Equation \ref{eqn:big_matrix}, that relates the external forces, $N$ and $M$, to the internal stresses and strains, $\varepsilon^0$ and $\kappa$.
\begin{align} \label{eqn:big_matrix}
	\begin{bmatrix}
		\textbf{N}\\
		\textbf{M}
	\end{bmatrix}
	&=
	\begin{bmatrix}
		\textbf{A}&\textbf{B}\\
		\textbf{B}&\textbf{D}
	\end{bmatrix}
	\begin{bmatrix}
		\boldsymbol{\upvarepsilon}^0\\
		\boldsymbol{\upkappa}
	\end{bmatrix}
\end{align}
The \textbf{A}-\textbf{B}-\textbf{B}-\textbf{D} matrix is called the laminate stiffness matrix. It is a \numproduct{6x6} matrix---each element itself is a \numproduct{3x3} matrix. $\boldsymbol{\upvarepsilon}^0$ and $\boldsymbol{\upkappa}$ can be further broken down as shown below:
\begin{align*}
	\begin{bmatrix}
		\boldsymbol{\upvarepsilon}^0\\
		\boldsymbol{\upkappa}
	\end{bmatrix}
	&=
	\begin{bmatrix}
		\varepsilon_x^0\\
		\varepsilon_y^0\\
		\gamma_{xy}^0\\
		\kappa_x\\
		\kappa_y\\
		\kappa_{xy}\\
	\end{bmatrix}
\end{align*}
Furthermore, the \textbf{A}-\textbf{B}-\textbf{B}-\textbf{D} matrix can be inverted to get Equation \ref{eqn:inverted_big_matrix}:
\begin{align} \label{eqn:inverted_big_matrix}
	\begin{bmatrix}
		\boldsymbol{\upvarepsilon}^0\\
		\boldsymbol{\upkappa}
	\end{bmatrix}
	&=
	\begin{bmatrix}
		\textbf{a}&\textbf{b}\\
		\textbf{c}&\textbf{d}
	\end{bmatrix}
	\begin{bmatrix}
		\textbf{N}\\
		\textbf{M}
	\end{bmatrix}
\end{align}
where \textbf{a}, \textbf{b}, \textbf{c}, and \textbf{d} are defined below:
\begin{align*}
	\textbf{a}&=\textbf{A}^{-1}+\textbf{A}^{-1}\textbf{B}(\textbf{D}^*)^{-1}\textbf{B}\textbf{A}^{-1}\\
	\textbf{b}&=-\textbf{A}^{-1}\textbf{B}(\textbf{D}^*)^{-1}\\
	\textbf{c}&=\textbf{B}^T\\
	\textbf{d}&=(\textbf{D}^*)^{-1}\\
	\textbf{D}^*&=\textbf{D}-\textbf{B}\textbf{A}^{-1}\textbf{B}
\end{align*}
For symmetric laminates, $\textbf{a}=\textbf{A}^{-1}$, $\textbf{B}=\textbf{b}=\textbf{c}=0$, and $\textbf{d}=\textbf{D}^{-1}$. By examining Equation \ref{eqn:big_matrix}, it is clear that \textbf{A} is not related to the twisting curvature, $\kappa_{xy}$. Only \textbf{B} and \textbf{D} are related to the twisting curvature. But since $\textbf{B}=0$, only \textbf{D} affects $\kappa_{xy}$, and because $\textbf{d}=\textbf{D}^{-1}$, \textbf{d} also effects $\kappa_{xy}$.

Therefore, of the variables \textbf{A}, \textbf{B}, \textbf{D}, \textbf{a}, \textbf{b}, \textbf{c}, and \textbf{d}, only \textbf{D} and \textbf{d} have any effect on the twisting curvature, $\kappa_{xy}$ in a symmetrical composite.

\textbf{Design Process}
In the design process, the goal was to find a layup design that is maximally resistant to torsional bending, \ie maximize $\lvert\kappa_{xy}\rvert$.

\begin{enumerate}
	\item For our first design, we chose a very natural $[\num{0}/\num{90}]_\mathrm{s}$ design which resulted in $\kappa_{xy}=\qty{1.37e-17}{m^{-1}}$. Based on our hypothesis, we assumed correctly this would be a very bad design, but it provided a good starting point from which to make iterations upon our design. It was clear the lack of layers oriented between \ang{0} and \ang{90} was causing our design to have hardly any resistance to torsion.
	\item For our second design, we chose $[\num{0}/\num{30}/\num{60}/\num{90}]_\mathrm{s}$ which resulted in $\kappa_{xy}=\qty{-0.0761}{m^{-1}}$. This is a significant improvement over the first design, but is still far from optimal. We chose this design to determine whether or not the addition of more layers between \ang{0} and \ang{90} would increase the resistance to torsion. It did increase $\kappa_{xy}$, but it was unclear whether or not the increase was due to the addition of more layers or the addition to layers in between \ang{0} and \ang{90}.
	\item For our third design, we chose $[\num{0}/\num{45}/\num{90}]_\mathrm{s}$ which resulted in $\kappa_{xy}=\qty{-0.0864}{m^{-1}}$. Our reasoning for this was to see if reducing the number of composite layers increased the resistance to bending. The result was that, yes, reducing the number of composite layers did increase the resistance to torsional bending, but only nominally. Based on our hypothesis, we expected the value of $kappa_{xy}$ to become smaller relative to the last design, but our hypothesis was proven wrong. At this point in the design process, we are led to conclude that less composite layers is better for increasing the resistance to torsional bending.
	\item For our fourth design, we chose $[\num{45}]_\mathrm{s}$ which resulted in $\kappa_{xy}=\qty{-28.3}{m^{-1}}$. We chose this design since we knew that $[\num{0}/\num{90}]_\mathrm{s}$ was very bad at reducing torsional bending effects and because we concluded that less composite layers were better at increasing torsional resistance. This partially confirmed our hypothesis that having layup angles in between \ang{0} and \ang{90} would increase resistance to torsional bending. We also noted that a layup design of $[\pm\num{45}]_\mathrm{s}$ resulted in the same magnitude of $\kappa_{xy}$ just with opposite signs.
	\item For our fifth design, we chose $[\num{27}]_\mathrm{s}$ which resulted in $\kappa_{xy}=\qty{-46.6}{m^{-1}}$. We came to this design by first increasing the previous design angle by \ang{1}. When we noted that this decreased the magnitude of $\kappa_{xy}$, we tried decreasing the layup angle which resulted in the magnitude of $\kappa_{xy}$ increasing. We decreased the layup angle in steps of \ang{1} degree until the magnitude of $\kappa_{xy}$ started to decrease again. This led us to the optimal value of $\kappa_{xy}$ being $\pm\qty{46.6}{m^{-1}}$ based on a layup design of $[\pm\num{27}]_\mathrm{s}$. This somewhat counter-intuitive result contradicts the majority of our hypothesis. We suspect this particular result is also very dependent on the geometric properties of the beam in question.
\end{enumerate}

\section{Conclusion} \label{conclusion}
Our hypothesis was partially correct. We accurately predicted that layup angles of \ang{0} and \ang{90} had a worse affect on $\kappa_{xy}$. We did not predict, however, that the largest $\kappa_{xy}$ would occur with only two layers. We were correct that maximum angle would be between \ang{0} and \ang{90}, but it was slightly below our expected range of \ang{30} to \ang{60} at \ang{27}.

In general we learned that the layup angle has a significant affect on the material's resistance to torsion. Additionally, we learned that adding more layers does not necessarily increase the resistance to torsion or any other desired material property.
\end{document}
